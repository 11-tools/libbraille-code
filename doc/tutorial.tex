\documentclass[12pt,a4paper]{article}
%%\usepackage[T1]{fontenc}
%%\usepackage[latin1]{inputenc}
%%\usepackage{babel}
%%\usepackage{epsfig}

\author{S\'ebastien Sabl\'e <sable@users.sourceforge.net>}
\title{Libbraille developer tutorial}

\begin{document}
\maketitle
\tableofcontents

This document explains how to develop using libbraille in the
C/C++ languages. It is mostly intended for developers and not for
simple users.

In this tutorial we will write a simple test program step by step that
uses most functionalities of the Braille library. The complete test
program is shown in annex and can be found in the test directory of
the source distribution.

\section{Generalities}

Interaction with the Braille library is done through functions
starting with the {\tt braille\_} prefix. In order to use those
functions, you need to include a {\tt braille.h} header.

\begin{verbatim}
/* test.c - Test program and simple example */

#include <stdio.h>
#include <stdlib.h>
#include "braille.h" // the required header
\end{verbatim}   

Most libbraille functions can return an error, usually by returning
0. In this case, it is possible to get a string describing precisely
the problem by calling the {\tt braille\_geterror()} function.

\begin{verbatim}
if(!braille_init())
  {
    fprintf(stderr, "Error initialising libbraille: %s\n", braille_geterror());
    return 1;
  }
\end{verbatim}   

\section{Configuration}

In order to run, libbraille must use a driver specific to the
connected Braille display; it also needs to know what Braille table
should be used to convert ASCII characters to Braille, as well as the
port to which the display is connected.

Parameters like the Braille display or port can usally be autodetected
so that it is not necessary to specify them. All parameters can also
be configured through a configuration file [TODO: or through the
windows registry on windows platforms].

However, an application may want to provide its own configuration
system to configure libbraille parameters. In order to permit that,
libbraille provides some functions to configure and request the
current configuration.

\subsection{Driver configuration}

The list of available drivers can be obtained with the following
functions:
\begin{description}
\item[{\tt braille\_drivernum()}] returns the number of available drivers
\item[{\tt braille\_drivername(i)}] returns the name of driver number
i
\item[{\tt braille\_drivermodels(i)}] returns a string describing the
models supported by driver number i
\end{description}

The following example shows how those functions can be used:
\begin{verbatim}
for(i = 0; i < braille_drivernum(); i++)
  {
    printf("driver [%d]: %s, supporting models: %s", i, braille_drivername(i),
	     braille_drivermodels(i));
    printf("\n");
  }
\end{verbatim}   

Once the name of the driver to use has been selected, it can be
configured in libbraille by using the {\tt braille\_config} like in
this example where ``once'' is the name of a driver:

\begin{verbatim}
braille_config(BRL_DRIVER, "once");
\end{verbatim}

Note that there is a special driver named ``auto''. This driver can be
used when the Braille display should be autodetected.

\subsection{Braille table configuration}

The list of available Braille tables can be obtained with the
following functions:
\begin{description}
\item[{\tt braille\_tablenum()}] returns the number of available
tables
\item[{\tt braille\_tablename(i)}] returns the name of table number i
\item[{\tt braille\_tabledesc(i)}] returns a string describing the
Braille table number i
\end{description}

The following example shows how those functions can be used:
\begin{verbatim}
for(i = 0; i < braille_tablenum(); i++)
  {
    printf("table [%d]: %s, corresponding to: %s", i, braille_tablename(i),
	     braille_tabledesc(i));
    printf("\n");
  }
\end{verbatim}

Once the name of the table to use has been selected, it can be
configured in libbraille by using the {\tt braille\_config} like in
this example where ``french.tbl'' is the name of a table:
\begin{verbatim}
braille_config(BRL_TABLE, "french.tbl");
\end{verbatim}

\subsection{Misc configuration}

The {\tt braille\_config} function can be used to configure other
parameters by changing the first argument:
\begin{description}
\item[{\tt BRL\_DEVICE}] to configure the port where the display is connected
\item[{\tt BRL\_PATH}] (in windows) to configure the path where
  libbraille have been installed
\item[{\tt BRL\_PATHCONF}] (unix only) to configure the path where
  the configuration file can be found
\item[{\tt BRL\_PATHDRV}] to configure the path where additionnal
  drivers can be found
\item[{\tt BRL\_PATHTBL}] to configure the path where additionnal
  tables can be found
\end{description}

A complete example:
\begin{verbatim}
braille_config(BRL_DEVICE, "COM1");
braille_config(BRL_DRIVER, "auto");
braille_config(BRL_TABLE, "foo.tbl");
braille_config(BRL_PATH, "/foo/bar/");
\end{verbatim}

\section{Initialisation}

The {\tt braille\_init} function should be called right after
configuration, before any other function of the library. It will parse
the configuration file, autodetect the Braille display if necessary,
load the correct driver and Braille table then initialise and
configure the Braille display.

\begin{verbatim}
if(!braille_init())
  {
    fprintf(stderr, "Error initialising libbraille: %s\n", braille_geterror());
    return 1;
  }
\end{verbatim}   

\section{Query configuration}

Once the library has been initialised, it is possible to query the
configuration that is used by using the {\tt braille\_config} function
with a specific parameter:
\begin{description}
\item[{\tt BRL\_DEVICE}] to query the port where the display is connected
\item[{\tt BRL\_DRIVER}] to query the name of the Braille display driver
\item[{\tt BRL\_TERMINAL}] to query for the model of Braille display
which is connected
\item[{\tt BRL\_TABLE}] to query the name of the Braille table
\item[{\tt BRL\_PATH}] to query the path where libbraille data files
have been found
\item[{\tt BRL\_PATHCONF}] (unix only) to query the path where the
  configuration file can be found
\item[{\tt BRL\_PATHDRV}] to query the path where additionnal
  drivers can be found
\item[{\tt BRL\_PATHTBL}] to query the path where additionnal
  tables can be found
\item[{\tt BRL\_VERSION}] to query the version of libbraille
\end{description}

The {\tt braille\_size()} function returns the number of cells of
Braille display. The {\tt braille\_statussize()} function returns the
number of additionnal cells for the status area.

\begin{verbatim}
printf("Libbraille version %s initialised correctly "
       "with driver %s using terminal %s with %d cells on device %s\n",
       braille_info(BRL_VERSION), braille_info(BRL_DRIVER),
       braille_info(BRL_TERMINAL), braille_size(),
       braille_info(BRL_DEVICE));
\end{verbatim}

\section{Displaying a Simple String}

The easiest way to write something on the Braille display is to use
the {\tt braille\_display} function. It must be called with a string
terminated by {\tt NULL} and will display that string on the display.

\begin{verbatim}
braille_display("test_libbraille started");
\end{verbatim}   

\section{Advanced dots displaying}

There is a more complex function to display something, when a better
control of what is displayed is necessary, for example when displaying
something other than text. What will be displayed is a combination of
text and a filter that directly manipulates dots.

First you set the text with {\tt braille\_write}. Contrarly to {\tt
braille\_display}, this function does not take a {\tt NULL}
terminated string as parameter, but a pointer to some chars and a
number indicating the number of chars to read.
   
Then you use {\tt braille\_filter} to raise some dots. The first
argument of {\tt braille\_filter} is an unsigned char corresponding to
which dots have to be activated. The second argument is the number of
the cell which has to be modified, starting at 0.

The dots 1, 2, 3, 4, 5, 6 and 7 correspond respectively to bits 0, 1,
2, 3, 4, 5, 6 and 7 of the unsigned char.  For example, a cursor made
of dots 7 and 8 corresponds to the binary number 00000011. There is a
{\tt BRAILLE} macro that makes it simple to get the correct value. It
just takes as parameters the state of each dot.

Finally, {\tt braille\_render} is called. This function filters the
text given by {\tt braille\_write} with the filter defined through
{\tt braille\_filter} and send the data to the Braille display.

The following example display some text with a cursor under the first
character:
\begin{verbatim}
{  
  char *hello_msg = "more complex test";

  braille_write(hello_msg, strlen(hello_msg)); 
  braille_filter(BRAILLE(0, 0, 0, 0, 0, 0, 1, 1), 0);
  braille_render();
}   
\end{verbatim}

\section{Typing with the Braille Keyboard}

Most Braille displays have some keys like cursors or function keys to
interact with the computer. Some even feature a Braille keyboard to
input text. Libbraille makes it possible to get an event when some of
those keys are pressed by using the {\tt braille\_read} function.

\subsection{Setting the timeout}

Depending on the application, it is sometime necessary to control how
much time a function can take before returning. There is a function
called {\tt braille\_timeout} to configure how the {\tt braille\_read}
function should behave:
\begin{description}
\item[blocking] in this mode, {\tt braille\_read} will wait forever
that a key is pressed on the display. This mode is set by passing -1
to {\tt braille\_timeout}.
\item[immediat return] in this mode, {\tt braille\_read} will test if
a key has been pressed and return immediately after. This mode is set
by passing 0 to {\tt braille\_timeout}.
\item[time limited] in this mode, {\tt braille\_read} will wait at
most during a given time that a key is pressed on the display. This
mode is set by passing a time expressed in milliseconds to
{\tt braille\_timeout}.
\end{description}

\subsection{Reading data}

When a key is pressed or at the time limit, the {\tt braille \_read}
function returns a structure of type {\tt brl\_key} describing the
event that happened.

The structure has an attribute named {\tt type} describing the type of
key pressed. It can have a few values:
\begin{description}
\item[{\tt BRL\_NONE}] in this case, no key has been pressed. The function
returned because of a timeout
\item[{\tt BRL\_CURSOR}] a ``cursor routing'' key has been
pressed. Those are the keys which are just above the Braille cells. In
this case, the {\tt code} attribute contains the number starting at 0
of the cell for which the cursor routing key has been pressed.
\item[{\tt BRL\_CMD}] a ``function'' key has been pressed. Those are
keys with a special function like validate, read further on the right
or move at the top of the page. The {\tt code} attribute contains a
code depending on the function key. There are many codes which can be
found in the {\tt braille.h} header file.
\item[{\tt BRL\_KEY}] the user has typed some ASCII characters on the
display. The {\tt code} attribute gives the ASCII value [TODO: and
{\tt unicode} provides the unicode value] of the character.
\end{description}

\begin{verbatim}
/* we want the braille_read function to wait forever while no key
   has been pressed */
braille_timeout(-1);

/* get the keys pressed on the display */
while(1)
  {
    signed char status;
    brl_key key;
    
    status = braille_read(&key);
    if(status == -1)
      {
        printf("error in braille_read: %s",
	       braille_geterror());
      }
    else if(status)
      {
     printf("Read key with type %d and code %d\n",
            key.type, key.code);
     switch(key.type)
       {
        case BRL_NONE:
          break;
        case BRL_CURSOR:
          printf("cursor: %d\n", key.code);
          break;
        case BRL_CMD:
          printf("command: ");
          switch(key.code)
    	{
    	case BRLK_HOME:
    	  printf("home\n");
    	  break;
    	case BRLK_BACKWARD:
    	  braille_display("backward");
    	  printf("reading further left on the display\n");
    	  break;
    	case BRLK_FORWARD:
    	  braille_display("forward");
    	  printf("reading further right on the display\n");
    	  break;
    	default:
    	  printf("unknown cmd\n");
    	  break;
    	}
          break;
        case BRL_KEY:
          printf("braille: 0x%x\n", key.braille);
          printf("code: %d ou 0x%x\n", key.code, key.code);
          printf("char: %c\n", key.code);
          break;
        default:
          printf("unknown type %d\n", key.type);
        }
    }
  }
\end{verbatim}

\section{Stopping the Library}

Finally, the {\tt braille\_close} function must always be called when
closing the Braille library. It will unload the driver, free resources
and close the library.

\begin{verbatim}
braille_close();
printf("Leaving test_libbraille\n");
\end{verbatim}

\section{Compiling}

Libbraille is made of a shared library. In order to use it you need to
link the library to your program.

For Visual C++ developpers, you need to add the libbraille-1.lib file
to your project. An example Visual C++ project and the .lib file is
available on the download page of the web site as
libbraille-dev-x.x.x-VC6.zip

For unix developpers, linking the library is just a matter of
adding the {\tt-lbraille} option at compile time.
For example, compiling the previous test code would be :
\begin{verbatim}
    cc test.c -o test -lbraille
\end{verbatim}

\section*{Annex}
\begin{verbatim}
/*
 * test_libbraille.c - Test program and simple example
 */

#include <stdio.h>
#include <stdlib.h>
#include <string.h>
#include "braille.h"

int 
main(int argc, char *argv[])
{
  int i;

  printf("Starting test_libbraille\n");

  for(i = 0; i < braille_drivernum(); i++)
    {
      printf("driver [%d]: %s, supporting models: %s", i, braille_drivername(i),
	     braille_drivermodels(i));
      printf("\n");
    }

  for(i = 0; i < braille_tablenum(); i++)
    {
      printf("table [%d]: %s, corresponding to: %s", i, braille_tablename(i),
	     braille_tabledesc(i));
      printf("\n");
    }

  /* uncomment to config some values without using the configuration
     file */
  /* braille_config(BRL_DEVICE, "COM1"); */
  /* braille_config(BRL_DRIVER, "eco"); */
  /* or braille_config(BRL_DRIVER, "auto"); */
  /* braille_config(BRL_TABLE, "foo.tbl"); */
  /* braille_config(BRL_PATH, "/foo/bar/"); */
  /* ... */

  if(!braille_init())
    {
      fprintf(stderr, "Error initialising libbraille: %s\n",
                      braille_geterror());
      
      fprintf(stderr, "Testing fake driver\n");
      braille_config(BRL_DRIVER, "none");
      if(!braille_init())
	{
	  fprintf(stderr, "Error initialising libbraille: %s\n",
                          braille_geterror());
	  return 1;
	}
    }

  printf("Libbraille version %s initialised correctly " \
	 "with driver %s using terminal %s with %d cells on device %s\n",
	 braille_info(BRL_VERSION), braille_info(BRL_DRIVER),
	 braille_info(BRL_TERMINAL), braille_size(),
	 braille_info(BRL_DEVICE));

  /* we want the braille_read function to wait forever while no key
     has been pressed */
  // braille_timeout(-1);

  /* we want the braille_read function to wait at most 1s while no key
     has been pressed */
  braille_timeout(1000);

  printf("displaying string\n");
  /* simple way to display something */
  braille_display("test_libbraille started");

  /* complex and powerful way to display something */
  {
    char *hello_msg = "test_libbraille started";

    braille_write(hello_msg, strlen(hello_msg)); 
    braille_filter(BRAILLE(0, 0, 0, 0, 0, 0, 1, 1), 0);
    braille_filter(BRAILLE(0, 0, 0, 0, 0, 0, 1, 1), 23);
    braille_render();
  }

  /* Displaying a few things on the status cells */
  if(braille_statussize() > 0)
    braille_statusdisplay("abc");

  /* get the keys pressed on the display */
  while(1)
    {
      signed char status;
      brl_key key;
      
      status = braille_read(&key);
      if(status == -1)
	{
	  printf("error in braille_read: %s", braille_geterror());
	}
      else if(status)
	{
	  printf("Read key with type %d and code %d\n", key.type, key.code);
	  switch(key.type)
	    {
	    case BRL_NONE:
	      break;
	    case BRL_CURSOR:
	      printf("cursor: %d\n", key.code);
	      break;
	    case BRL_CMD:
	      printf("command: ");
	      switch(key.code)
		{
		case BRLK_HOME:
		  printf("home\n");
		  break;
		case BRLK_BACKWARD:
		  braille_display("backward");
		  printf("reading further left on the display\n");
		  break;
		case BRLK_FORWARD:
		  braille_display("forward");
		  printf("reading further right on the display\n");
		  break;
		default:
		  printf("unknown cmd\n");
		  break;
		}
	      break;
	    case BRL_KEY:
	      printf("braille: 0x%x\n", key.braille);
	      printf("code: %d ou 0x%x\n", key.code, key.code);
	      printf("char: %c\n", key.code);
	      break;
	    default:
	      printf("unknown type %d\n", key.type);
	    }
	}
    }

  /* now we stop everything */
  braille_close();

  printf("Leaving test_libbraille\n");
  return 0;
}
\end{verbatim}
\end{document}
